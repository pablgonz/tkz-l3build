% tkz-obj-eu-draw-circles.tex
% Copyright 2020  Alain Matthes
% This work may be distributed and/or modified under the
% conditions of the LaTeX Project Public License, either version 1.3
% of this license or (at your option) any later version.
% The latest version of this license is in
%   http://www.latex-project.org/lppl.txt
% and version 1.3 or later is part of all distributions of LaTeX
% version 2005/12/01 or later.
%
% This work has the LPPL maintenance status “maintained”.
% 
% The Current Maintainer of this work is Alain Matthes.

%  utf8 encoding
\def\fileversion{3.05c}
\def\filedate{2020/03/03} 
\typeout{2020/03/03 3.05c  tkz-obj-eu-draw-circles.tex} 
\makeatletter 
%for compatibility
%<--------------------------------------------------------------------------–>
\def\tkzDefCircleR(#1,#2){% 
\begingroup 
  \edef\tkzLengthResult{\fpeval{round(#2,5)}}
  \global\let\tkzLengthResult\tkzLengthResult  
  \tkzRenamePoint(#1){tkzPointResult}   
\endgroup
} 
%<--------------------------------------------------------------------------–>
%              Through  Circle center and point
%<--------------------------------------------------------------------------–>
\def\tkzDefCircleThrough(#1,#2){% 
\begingroup
  \tkzCalcLength(#1,#2)
  \tkzRenamePoint(#1){tkzPointResult}
\endgroup
}    
%<--------------------------------------------------------------------------–>
%              Diameter Circle
%<--------------------------------------------------------------------------–>
\def\tkzDefCircleD(#1,#2){% 
\begingroup
  \tkzDefMidPoint(#1,#2)
  \tkzCalcLength(#1,tkzPointResult)
\endgroup
} 

\def\tkz@numdc{0}
\pgfkeys{/tkzdrawc/.cd,
      through/.code                 =   \def\tkz@numdc{0},
      R/.code                       =   \def\tkz@numdc{1},
      diameter/.code                =   \def\tkz@numdc{2},
      circum/.code                  =   \def\tkz@numdc{3},
      in/.code                      =   \def\tkz@numdc{4},
      ex/.code                      =   \def\tkz@numdc{5},
      euler/.code                   =   \def\tkz@numdc{6},
      nine/.code                    =   \def\tkz@numdc{6},
      apollonius/.code              =   \def\tkz@numdc{7},
      orthogonal from/.code args    =  {#1}{\def\tkz@ptfrom{#1},
                                        \def\tkz@numdc{8}}, 
      orthogonal through/.code args =  {#1 and #2}{\def\tkz@ptone{#1}
                                                   \def\tkz@pttwo{#2} 
                                                   \def\tkz@numdc{9}}, 
      K/.store in                   = \tkz@koeff,% apollonius
      through,
      K                      =   1,
   /tkzdrawc/.search also={/tikz}
} 
%<--------------------------------------------------------------------------–>
%<--------------------------------------------------------------------------–>
%            Drawing a circle                                                >
%<--------------------------------------------------------------------------–>
\def\tkzDrawCircle{\pgfutil@ifnextchar[{\tkz@DrawCircle}{\tkz@DrawCircle[]}}
\def\tkz@DrawCircle[#1](#2){%     
\begingroup 
\pgfqkeys{/tkzdrawc}{#1}  
\ifcase\tkz@numdc%
    \tkzDefCircleThrough(#2)
  \or% 1
    \tkzDefCircleR(#2) 
  \or% 2
    \tkzDefCircleD(#2)  
  \or% 3
      \tkzDefCircumCircle(#2) 
  \or% 4
   \tkzDefInCircle(#2)
  \or% 4
   \tkzDefExCircle(#2)
  \or% 5
   \tkzDefEulerCircle(#2) 
  \or% 6
   \tkzDefApolloniusCircle(#2) 
  \or% 7
   \tkzDefOrthogonalCircle(#2,\tkz@ptfrom) 
  \or% 8
   \tkzDefOrthoThroughCircle(#2,\tkz@ptone,\tkz@pttwo)  
 \fi
   \draw[line style,/tkzdrawc/.cd,#1]
                  (tkzPointResult) circle (\tkzLengthResult pt);      
\endgroup
}

%<--------------------------------------------------------------------------–>
\def\tkz@multicircles#1 #2\@nil{% 
\protected@edef\tkz@temp{
\noexpand \tkzDrawCircle[\tkz@optcircle](#1)}\tkz@temp% 
\def\tkz@nextArg{#2}%
\ifx\tkzutil@empty\tkz@nextArg
     \let\next\@gobble
\fi
\next#2\@nil
}
%<--------------------------------------------------------------------------–>
\def\tkzDrawCircles{\pgfutil@ifnextchar[{\tkz@DrawCircles}{\tkz@DrawCircles[]}} 
\def\tkz@DrawCircles[#1](#2){%
\xdef\tkz@optcircle{#1} 
\begingroup
   \let\next\tkz@multicircles
   \next#2 \@nil %    
\endgroup     
}% 
%<--------------------------------------------------------------------------–>
%<--------------------------------------------------------------------------–>  
% #2 #3 rayon
\def\tkz@numdsc{0}
\pgfkeys{/tkzdrawsc/.cd,
       through/.code         =  \def\tkz@numdsc{0},
       diameter/.code        =  \def\tkz@numdsc{1},
       swap/.is if           =  tkz@sawp@sc,
       swap/.default         =  true,
       swap                  =  false,
       through,
         /tkzdrawsc/.search also={/tikz}
   }  
\def\tkzDrawSemiCircle{\pgfutil@ifnextchar[{\tkz@DrawSemiCircle}{%
                                    \tkz@DrawSemiCircle[]}}
\def\tkz@DrawSemiCircle[#1](#2){% 
\begingroup 
\pgfqkeys{/tkzdrawsc}{#1}  
\ifcase\tkz@numdsc%
   \tkzDrawSemiCircleThrough(#2)
\or%
   \tkzDrawSemiCircleDiameter(#2)
\fi
 \tkzDrawArc[#1](\tkz@Center,\tkz@FirstPoint)(\tkz@SecondPoint)
\endgroup
}
%<--------------------------------------------------------------------------–> 
\def\tkzDrawSemiCircleThrough(#1,#2){%
   \tkzDefPointBy[symmetry=center #1](#2)
         \pgfnodealias{tkz@pt}{tkzPointResult}
   \def\tkz@Center{#1}
   \iftkz@sawp@sc
         \gdef\tkz@FirstPoint{tkz@pt}
      \gdef\tkz@SecondPoint{#2}
   \else
         \gdef\tkz@FirstPoint{#2}
      \gdef\tkz@SecondPoint{tkz@pt}
   \fi
}
%<--------------------------------------------------------------------------–>
\def\tkzDrawSemiCircleDiameter(#1,#2){%
  \tkzDefMidPoint(#1,#2) 
  \pgfnodealias{tkz@Center}{tkzPointResult}
  \def\tkz@Center{tkz@Center}
  \iftkz@sawp@sc
      \def\tkz@FirstPoint{#1}
      \def\tkz@SecondPoint{#2}
   \else
      \def\tkz@FirstPoint{#2}
      \def\tkz@SecondPoint{#1}
   \fi
}

%<---------------------------- Fill Circle  --------------------------------–>
\def\tkz@numfc{0}
\pgfkeys{/fillcircle/.cd, radius/.code      =      \def\tkz@numfc{0},
                       R/.code           =      \def\tkz@numfc{1},
                    radius, 
  /fillcircle/.search also={/tikz}
}
\def\tkzFillCircle{\pgfutil@ifnextchar[{\tkz@FillCircle}{\tkz@FillCircle[]}}
\def\tkz@FillCircle[#1](#2,#3){%
\begingroup      
\pgfqkeys{/fillcircle}{#1}
\ifcase\tkz@numfc%
 % first case 0
    \tkzCalcLength(#2,#3)
   \fill[/fillcircle/.cd,#1] (#2) circle (\tkzLengthResult pt);%  
  \or% 1
   \fill[/fillcircle/.cd,#1] (#2) circle (#3);%  
  \fi    
\endgroup
}
%<----------------------- Out Clip Circle  ---------------------------------–>
\def\tkz@numocc{0}
\pgfkeys{/clipoutcircle/.cd, radius/.code =  \def\tkz@numocc{0},
                         R/.code      =   \def\tkz@numocc{1},
                      radius}
%<--------------------------------------------------------------------------–>
\def\tkzClipOutCircle{\pgfutil@ifnextchar[{\tkz@ClipOutCircle}{%
                                        \tkz@ClipOutCircle[]}}
\def\tkz@ClipOutCircle[#1](#2,#3){%    
\pgfqkeys{/clipoutcircle}{#1}
\ifcase\tkz@numocc
     \tkzCalcLength(#2,#3)
     \clip[tkzreverseclip]  (#2) circle (\tkzLengthResult pt);  
  \or% 1
     \clip[tkzreverseclip]  (#2) circle (#3);  
  \fi
} 
%<--------------------------- Clip Circle  ---------------------------------–>
\def\tkz@numcc{0}
\pgfkeys{/tkzclipc/.cd,    
         through/.code    =  \def\tkz@numcc{0},
         R/.code          =  \def\tkz@numcc{1},
         through}
%<--------------------------------------------------------------------------–>
\def\tkzClipCircle{\pgfutil@ifnextchar[{\tkz@ClipCircle}{\tkz@ClipCircle[]}}

\def\tkz@ClipCircle[#1](#2,#3){%    
\pgfqkeys{/tkzclipc}{#1}
\ifcase\tkz@numcc
   \tkzCalcLength(#2,#3)
   \clip  (#2) circle (\tkzLengthResult pt);
 \or% 1
   \clip  (#2) circle (#3);  
 \fi
} 
%<--------------------------- Label Circle  --------------------------------–>
% attention radius circle is defined by center and a point on the circle
% R defined by center and the value of the radius
\def\tkz@numlc{0}
\pgfkeys{/tkzlabelc/.cd,
          through/.code    = \def\tkz@numlc{0},
          R/.code          = \def\tkz@numlc{1},
          through,
          /tkzlabelc/.search also={/tikz}
}

\def\tkzLabelCircle{\pgfutil@ifnextchar[{\tkz@LabelCircle}{%
                                         \tkz@LabelCircle[]}}
% [option]  (#2,#3) #2 center #3 soit un point du cercle soit le radius
%          #4 angle #5 the label
\def\tkz@LabelCircle[#1](#2,#3)(#4)#5{%
\begingroup      
\pgfqkeys{/tkzlabelc}{#1}
\ifcase\tkz@numlc
   \tkzURotateAngle(#2,#4)(#3)
   \node[/tkzlabelc/.cd,#1] at (tkzPointResult) {#5};    
\or% 1
  \path (#2)--++(#3,0) coordinate (tkzPointResult);
  \tkzURotateAngle(#2,#4)(tkzPointResult)
   \node[/tkzlabelc/.cd,#1] at (tkzPointResult) {#5};   
\fi    
\endgroup
}
%<--------------------------------------------------------------------------–> 
 \makeatother 
\endinput