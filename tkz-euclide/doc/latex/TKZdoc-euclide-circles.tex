\section{The Circles}

Among the following macros, one will allow you to draw a circle, which is not a real feat. To do this, you will need to know the center of the circle and either the radius of the circle or a point on the circumference. It seemed to me that the most frequent use was to draw a circle with a given centre passing through a given point. This will be the default method, otherwise you will have to use the \tkzname{R} option. There are a large number of special circles, for example the circle circumscribed by a triangle.

\begin{itemize}
  \item  I have created a first macro \tkzcname{tkzDefCircle} which allows, according to a particular circle, to retrieve its center and the measurement of the radius in cm. This recovery is done with the macros \tkzcname{tkzGetPoint} and \tkzcname{tkzGetLength};

 \item then a macro \tkzcname{tkzDrawCircle};

 \item then a macro that allows you to color in a disc, but without drawing the circle \tkzcname{tkzFillCircle};

 \item sometimes, it is necessary for a drawing to be contained in a disk, this is the role assigned to \tkzcname{tkzClipCircle};


 \item  it finally remains to be able to give a label to designate a circle and if several possibilities are offered, we will see here \tkzcname{tkzLabelCircle}.
\end{itemize}

\subsection{Characteristics of a circle: \tkzcname{tkzDefCircle}}

This macro allows you to retrieve the characteristics (center and radius) of certain circles.

\begin{NewMacroBox}{tkzDefCircle}{\oarg{local options}\parg{A,B} or \parg{A,B,C}}%
\tkzHandBomb\ Attention the arguments are lists of two or three points. This macro is either used in partnership with \tkzcname{tkzGetPoint} and/or \tkzcname{tkzGetLength} to obtain the center and the radius of the circle, or by using \tkzname{tkzPointResult} and \tkzname{tkzLengthResult} if it is not necessary to keep the results.

\medskip
\begin{tabular}{lll}%
\toprule
arguments           & example & explication                         \\
\midrule
\TAline{\parg{pt1,pt2} or \parg{pt1,pt2,pt3}}{\parg{A,B}} {$[AB]$ is radius $A$ is the center}
\bottomrule
\end{tabular}

\medskip
\begin{tabular}{lll}%
\toprule
options             & default & definition                         \\
\midrule
\TOline{through}      {through}{circle characterized by two points defining a radius}
\TOline{diameter}     {through}{circle characterized by two points defining a diameter}
\TOline{circum}       {through}{circle circumscribed of a triangle}
\TOline{in}           {through}{incircle a triangle}
\TOline{ex}           {through}{excircle of a  triangle}
\TOline{euler or nine}{through}{Euler's Circle}
\TOline{spieker}      {through}{Spieker Circle}
\TOline{apollonius}   {through}{circle of Apollonius}
\TOline{orthogonal}   {through}{circle of given centre orthogonal to another circle}
\TOline{orthogonal through}{through}{circle orthogonal circle passing through 2 points}
\TOline{K} {1}{coefficient used for a circle of Apollonius}
 \bottomrule
\end{tabular}

{In the following examples, I draw the circles with a macro not yet presented, but this is not necessary. In some cases you may only need the center or the radius.}
\end{NewMacroBox}

 \subsubsection{Example with a random point and  option \tkzname{through}}

\begin{tkzexample}[latex=7 cm,small]
 \begin{tikzpicture}[scale=1]
 \tkzDefPoint(0,4){A}
 \tkzDefPoint(2,2){B}
 \tkzDefMidPoint(A,B) \tkzGetPoint{I}
 \tkzDefRandPointOn[segment = I--B]
  \tkzGetPoint{C}
 \tkzDefCircle[through](A,C)
 \tkzGetLength{rACpt}
 \tkzpttocm(\rACpt){rACcm}
 \tkzDrawCircle(A,C)
 \tkzDrawPoints(A,B,C)
 \tkzLabelPoints(A,B,C)
 \tkzLabelCircle[draw,fill=orange,
          text width=3cm,text centered,
          font=\scriptsize](A,C)(-90)%
 {The radius measurement is:
  \rACpt pt i.e. \rACcm cm}
 \end{tikzpicture}
 \end{tkzexample}

 \subsubsection{Example with  option \tkzname{diameter}}
 It is simpler here to search directly for the middle of $[AB]$.
 \begin{tkzexample}[latex=7cm,small]
 \begin{tikzpicture}[scale=1]
    \tkzDefPoint(0,0){A}
    \tkzDefPoint(2,2){B}
    \tkzDefCircle[diameter](A,B)
    \tkzGetPoint{O}
    \tkzDrawCircle[blue,fill=blue!20](O,B)
    \tkzDrawSegment(A,B)
    \tkzDrawPoints(A,B,O)
    \tkzLabelPoints(A,B,O)
 \end{tikzpicture}
 \end{tkzexample}

 \subsubsection{Circles inscribed and circumscribed for a given triangle}
 You can also obtain the center of the inscribed circle and its projection on one side of the triangle with \tkzcname{tkzGetFirstPoint{I}} and \tkzcname{tkzGetSecondPoint{Ib}}.


\begin{tkzexample}[latex=7cm,small]
\begin{tikzpicture}[scale=1]
   \tkzDefPoint(2,2){A}
   \tkzDefPoint(5,-2){B}
   \tkzDefPoint(1,-2){C}
   \tkzDefCircle[in](A,B,C)
   \tkzGetPoint{I}    \tkzGetLength{rIN}
   \tkzDefCircle[circum](A,B,C)
   \tkzGetPoint{K}   \tkzGetLength{rCI}
   \tkzDrawPoints(A,B,C,I,K)
   \tkzDrawCircle[R,blue](I,\rIN pt)
   \tkzDrawCircle[R,red](K,\rCI pt)
   \tkzLabelPoints[below](B,C)
   \tkzLabelPoints[above left](A,I,K)
   \tkzDrawPolygon(A,B,C)
\end{tikzpicture}
\end{tkzexample}

 \subsubsection{Example with option \tkzname{ex}}
We want to define an excircle of a  triangle relatively to point $C$

\begin{tkzexample}[latex=8cm,small]
\begin{tikzpicture}[scale=.75]
	\tkzDefPoints{ 0/0/A,4/0/B,0.8/4/C}
	\tkzDefCircle[ex](B,C,A)
  \tkzGetPoint{J_c} \tkzGetLength{rc}
	\tkzDefPointBy[projection=onto A--C ](J_c)
  \tkzGetPoint{X_c}
  \tkzDefPointBy[projection=onto A--B ](J_c)
  \tkzGetPoint{Y_c}
  \tkzGetPoint{I}
  \tkzDrawPolygon[color=blue](A,B,C)
  \tkzDrawCircle[R,color=lightgray](J_c,\rc pt)
  % possible  \tkzDrawCircle[ex](A,B,C)
  \tkzDrawCircle[in,color=red](A,B,C)    \tkzGetPoint{I}
	\tkzDefPointBy[projection=onto A--C ](I)
  \tkzGetPoint{F}
	\tkzDefPointBy[projection=onto A--B ](I)
  \tkzGetPoint{D}
	\tkzDrawLines[add=0 and 2.2,dashed](C,A C,B)
	\tkzDrawSegments[dashed](J_c,X_c I,D  I,F J_c,Y_c)
	\tkzMarkRightAngles(A,F,I B,D,I J_c,X_c,A J_c,Y_c,B)
	\tkzDrawPoints(B,C,A,I,D,F,X_c,J_c,Y_c)
 	\tkzLabelPoints(B,A,J_c,I,D,X_c,Y_c)
 	\tkzLabelPoints[above left](C)
	\tkzLabelPoints[left](F)
\end{tikzpicture}
\end{tkzexample}

 \subsubsection{Euler's circle for a given triangle with option \tkzname{euler}}

We verify that this circle passes through the middle of each side.
\begin{tkzexample}[latex=8cm,small]
\begin{tikzpicture}[scale=.75]
   \tkzDefPoint(5,3.5){A}
   \tkzDefPoint(0,0){B} \tkzDefPoint(7,0){C}
   \tkzDefCircle[euler](A,B,C)
   \tkzGetPoint{E}  \tkzGetLength{rEuler}
   \tkzDefSpcTriangle[medial](A,B,C){M_a,M_b,M_c}
   \tkzDrawPoints(A,B,C,E,M_a,M_b,M_c)
   \tkzDrawCircle[R,blue](E,\rEuler pt)
   \tkzDrawPolygon(A,B,C)
   \tkzLabelPoints[below](B,C)
   \tkzLabelPoints[left](A,E)
\end{tikzpicture}
\end{tkzexample}

 \subsubsection{Apollonius circles for a given segment option \tkzname{apollonius}}

\begin{tkzexample}[latex=9cm,small]
\begin{tikzpicture}[scale=0.75]
  \tkzDefPoint(0,0){A}
  \tkzDefPoint(4,0){B}
  \tkzDefCircle[apollonius,K=2](A,B)
  \tkzGetPoint{K1}
  \tkzGetLength{rAp}
  \tkzDrawCircle[R,color = blue!50!black,
      fill=blue!20,opacity=.4](K1,\rAp pt)
  \tkzDefCircle[apollonius,K=3](A,B)
  \tkzGetPoint{K2}   \tkzGetLength{rAp}
  \tkzDrawCircle[R,color=red!50!black,
   fill=red!20,opacity=.4](K2,\rAp pt)
  \tkzLabelPoints[below](A,B,K1,K2)
  \tkzDrawPoints(A,B,K1,K2)
  \tkzDrawLine[add=.2 and 1](A,B)
\end{tikzpicture}
\end{tkzexample}

 \subsubsection{Circles exinscribed to a given triangle option \tkzname{ex}}
 You can also get the center and the projection of it on one side of the triangle.

 with \tkzcname{tkzGetFirstPoint\{Jb\}} and \tkzcname{tkzGetSecondPoint\{Tb\}}.

\begin{tkzexample}[latex=8cm,small]
\begin{tikzpicture}[scale=.6]
  \tkzDefPoint(0,0){A}
  \tkzDefPoint(3,0){B}
  \tkzDefPoint(1,2.5){C}
  \tkzDefCircle[ex](A,B,C) \tkzGetPoint{I}
    \tkzGetLength{rI}
  \tkzDefCircle[ex](C,A,B) \tkzGetPoint{J}
    \tkzGetLength{rJ}
  \tkzDefCircle[ex](B,C,A) \tkzGetPoint{K}
    \tkzGetLength{rK}
   \tkzDefCircle[in](B,C,A) \tkzGetPoint{O}
     \tkzGetLength{rO}
  \tkzDrawLines[add=1.5 and 1.5](A,B A,C B,C)
  \tkzDrawPoints(I,J,K)
  \tkzDrawPolygon(A,B,C)
  \tkzDrawPolygon[dashed](I,J,K)
  \tkzDrawCircle[R,blue!50!black](O,\rO)
  \tkzDrawSegments[dashed](A,K B,J C,I)
  \tkzDrawPoints(A,B,C)
  \tkzDrawCircles[R](J,{\rJ} I,{\rI} K,{\rK})
  \tkzLabelPoints(A,B,C,I,J,K)
\end{tikzpicture}
\end{tkzexample}

  \subsubsection{Spieker circle with option \tkzname{spieker}}
The  incircle of the medial triangle $M_aM_bM_c$ is the Spieker circle:

\begin{tkzexample}[latex=8cm, small]
\begin{tikzpicture}[scale=1]
  \tkzDefPoints{ 0/0/A,4/0/B,0.8/4/C}
   \tkzDefSpcTriangle[medial](A,B,C){M_a,M_b,M_c}
   \tkzDefTriangleCenter[spieker](A,B,C)
   \tkzGetPoint{S_p}
   \tkzDrawPolygon[blue](A,B,C)
   \tkzDrawPolygon[red](M_a,M_b,M_c)
   \tkzDrawPoints[blue](B,C,A)
   \tkzDrawPoints[red](M_a,M_b,M_c,S_p)
   \tkzDrawCircle[in,red](M_a,M_b,M_c)
   \tkzAutoLabelPoints[center=S_p,dist=.3](M_a,M_b,M_c)
   \tkzLabelPoints[blue,right](S_p)
   \tkzAutoLabelPoints[center=S_p](A,B,C)
\end{tikzpicture}
\end{tkzexample}


 \subsubsection{Orthogonal circle passing through two given points, option \tkzname{orthogonal through}}

\begin{tkzexample}[latex=8cm,small]
\begin{tikzpicture}[scale=1]
  \tkzDefPoint(0,0){O}
  \tkzDefPoint(1,0){A}
  \tkzDrawCircle(O,A)
  \tkzDefPoint(-1.5,-1.5){z1}
  \tkzDefPoint(1.5,-1.25){z2}
  \tkzDefCircle[orthogonal through=z1 and z2](O,A)
   \tkzGetPoint{c}
  \tkzDrawCircle[thick,color=red](tkzPointResult,z1)
  \tkzDrawPoints[fill=red,color=black,
  size=4](O,A,z1,z2,c)
  \tkzLabelPoints(O,A,z1,z2,c)
\end{tikzpicture}
\end{tkzexample}

\subsubsection{Orthogonal circle of given center}

\begin{tkzexample}[latex=7cm,small]
\begin{tikzpicture}[scale=.75]
  \tkzDefPoints{0/0/O,1/0/A}
  \tkzDefPoints{1.5/1.25/B,-2/-3/C}
  \tkzDefCircle[orthogonal from=B](O,A)
  \tkzGetPoints{z1}{z2}
  \tkzDefCircle[orthogonal from=C](O,A)
  \tkzGetPoints{t1}{t2}
  \tkzDrawCircle(O,A)
  \tkzDrawCircle[thick,color=red](B,z1)
  \tkzDrawCircle[thick,color=red](C,t1)
  \tkzDrawPoints(t1,t2,C)
  \tkzDrawPoints(z1,z2,O,A,B)
  \tkzLabelPoints(O,A,B,C)
\end{tikzpicture}
\end{tkzexample}

%<---------------------------------------------------------------------------->

\section{Draw, Label the Circles}
\begin{itemize}
 \item  I created a first macro  \tkzcname{tkzDrawCircle},

 \item then a macro that allows you to color a disc, but without drawing the circle. \tkzcname{tkzFillCircle},

 \item sometimes, it is necessary for a drawing to be contained in a disc,this is the role assigned to \tkzcname{tkzClipCircle},


 \item  It finally remains to be able to give a label to designate a circle and if several possibilities are offered, we will see here \tkzcname{tkzLabelCircle}.
\end{itemize}

\subsection{Draw a circle}
\begin{NewMacroBox}{tkzDrawCircle}{\oarg{local options}\parg{A,B}}%
\tkzHandBomb\ Attention you need only two points to define a radius or a diameter.  An additional option \tkzname{R} is available  to give a measure directly.

\medskip
\begin{tabular}{lll}%
\toprule
arguments           & example & explication                         \\
\midrule
\TAline{\parg{pt1,pt2}}{\parg{A,B}} {two points to define a radius or a diameter}
\bottomrule
\end{tabular}

\medskip
\begin{tabular}{lll}%
\toprule
options             & default & definition                         \\
\midrule
\TOline{through}{through}{circle with two points defining a radius}
\TOline{diameter}{through}{circle with two points defining a diameter}
\TOline{R} {through}{circle characterized by a point and the measurement of a radius}
 \bottomrule
\end{tabular}

\medskip
Of course, you have to add all the styles of \TIKZ\ for the tracings...
\end{NewMacroBox}

 \subsubsection{Circles and styles, draw a circle and color the disc}
 We'll see that it's possible to colour in a disc while tracing the circle.

\begin{tkzexample}[latex=7cm,small]
\begin{tikzpicture}
  \tkzDefPoint(0,0){O}
  \tkzDefPoint(3,0){A}
 % circle with centre O and passing through A
  \tkzDrawCircle[color=blue](O,A)
 % diameter circle $[OA]$
  \tkzDrawCircle[diameter,color=red,%
                 line width=2pt,fill=red!40,%
                 opacity=.5](O,A)
 % circle with centre O and radius = exp(1) cm
  \edef\rayon{\fpeval{0.25*exp(1)}}
  \tkzDrawCircle[R,color=orange](O,\rayon cm)
\end{tikzpicture}
\end{tkzexample}

\subsection{Drawing circles}
\begin{NewMacroBox}{tkzDrawCircles}{\oarg{local options}\parg{A,B C,D}}%
\tkzHandBomb\ Attention, the arguments are lists of two points. The circles that can be drawn are the same as in the previous macro. An additional option \tkzname{R} is available to give  a measure directly.

\medskip
\begin{tabular}{lll}%
\toprule
arguments           & example & explication                         \\
\midrule
\TAline{\parg{pt1,pt2 pt3,pt4 ...}}{\parg{A,B C,D}} {List of two points}
\bottomrule
\end{tabular}

\medskip
\begin{tabular}{lll}%
\toprule
options             & default & definition                         \\
\midrule
\TOline{through}{through}{circle with two points defining a radius}
\TOline{diameter}{through}{circle with two points defining a diameter}
\TOline{R} {through}{circle characterized by a point and the measurement of a radius}
 \bottomrule
\end{tabular}

\medskip
Of course, you have to add all the styles of \TIKZ\ for the tracings...
\end{NewMacroBox}

 \subsubsection{Circles defined by a triangle.}

\begin{tkzexample}[latex=9cm,small]
\begin{tikzpicture}
  \tkzDefPoint(0,0){A}
  \tkzDefPoint(2,0){B}
  \tkzDefPoint(3,2){C}
  \tkzDrawPolygon(A,B,C)
  \tkzDrawCircles(A,B B,C C,A)
  \tkzDrawPoints(A,B,C)
  \tkzLabelPoints(A,B,C)
\end{tikzpicture}
\end{tkzexample}

 \subsubsection{Concentric circles.}

\begin{tkzexample}[latex=7cm,small]
\begin{tikzpicture}
   \tkzDefPoint(0,0){A}
   \tkzDrawCircles[R](A,1cm A,2cm A,3cm)
   \tkzDrawPoint(A)
   \tkzLabelPoints(A)
\end{tikzpicture}
\end{tkzexample}

 \subsubsection{Exinscribed circles.}

\begin{tkzexample}[latex=7cm,small]
\begin{tikzpicture}[scale=1]
\tkzDefPoints{0/0/A,4/0/B,1/2.5/C}
\tkzDrawPolygon(A,B,C)
\tkzDefCircle[ex](B,C,A)
\tkzGetPoint{J_c} \tkzGetSecondPoint{T_c}
\tkzGetLength{rJc}
\tkzDrawCircle[R](J_c,{\rJc pt})
\tkzDrawLines[add=0 and 1](C,A C,B)
\tkzDrawSegment(J_c,T_c)
\tkzMarkRightAngle(J_c,T_c,B)
\tkzDrawPoints(A,B,C,J_c,T_c)
\end{tikzpicture}
\end{tkzexample}

\subsubsection{Cardioid}
Based on an idea by O. Reboux made with pst-eucl (Pstricks module) by D. Rodriguez.

 Its name comes from the Greek \textit{kardia (heart)}, in reference to its shape, and was given to it by Johan Castillon (Wikipedia).

\begin{tkzexample}[latex=7cm,small]
\begin{tikzpicture}[scale=.5]
  \tkzDefPoint(0,0){O}
  \tkzDefPoint(2,0){A}
  \foreach \ang in {5,10,...,360}{%
     \tkzDefPoint(\ang:2){M}
     \tkzDrawCircle(M,A)
   }
\end{tikzpicture}
\end{tkzexample}

\subsection{Draw a semicircle}
\begin{NewMacroBox}{tkzDrawSemiCircle}{\oarg{local options}\parg{A,B}}%

\medskip
\begin{tabular}{lll}%
\toprule
arguments           & example & explication                         \\
\midrule
\TAline{\parg{pt1,pt2}}{\parg{O,A} or\parg{A,B}} {radius or diameter}
\bottomrule
\end{tabular}

\medskip
\begin{tabular}{lll}%
\toprule
options             & default & definition                         \\
\midrule
\TOline{through}  {through}{circle characterized by two points defining a radius}
\TOline{diameter} {through}{circle characterized by two points defining a diameter}
\end{tabular}
\end{NewMacroBox}

\subsubsection{Use of \tkzcname{tkzDrawSemiCircle}}

\begin{tkzexample}[latex=6cm,small]
  \begin{tikzpicture}
     \tkzDefPoint(0,0){A} \tkzDefPoint(6,0){B}
     \tkzDefSquare(A,B) \tkzGetPoints{C}{D}
     \tkzDrawPolygon(B,C,D,A)
     \tkzDefPoint(3,6){F}
     \tkzDefTriangle[equilateral](C,D) \tkzGetPoint{I}
     \tkzDefPointBy[projection=onto B--C](I) \tkzGetPoint{J}
     \tkzInterLL(D,B)(I,J)  \tkzGetPoint{K}
     \tkzDefPointBy[symmetry=center K](B) \tkzGetPoint{M}
     \tkzDrawCircle(M,I)
     \tkzCalcLength(M,I)   \tkzGetLength{dMI}
     \tkzFillPolygon[color = red!50](A,B,C,D)
     \tkzFillCircle[R,color = yellow](M,\dMI pt)
     \tkzDrawSemiCircle[fill = blue!50!black](F,D)%
  \end{tikzpicture}
\end{tkzexample}


\subsection{Colouring a disc}
This was possible with the previous macro, but disk tracing was mandatory, this is no longer the case.

\begin{NewMacroBox}{tkzFillCircle}{\oarg{local options}\parg{A,B}}%
\begin{tabular}{lll}%
options             & default & definition                         \\
\midrule
\TOline{radius}  {radius}{two points define a radius}
\TOline{R} {radius}{a point and the measurement of a radius }
\bottomrule
\end{tabular}

\medskip
You don't need to put \tkzname{radius} because that's the default option. Of course, you have to add all the styles of \TIKZ\ for the plots.
\end{NewMacroBox}

 \subsubsection{Example from a sangaku}

\begin{tkzexample}[latex=7cm,small]
\begin{tikzpicture}
   \tkzInit[xmin=0,xmax = 6,ymin=0,ymax=6]
   \tkzDefPoint(0,0){B}  \tkzDefPoint(6,0){C}%
   \tkzDefSquare(B,C)    \tkzGetPoints{D}{A}
   \tkzClipPolygon(B,C,D,A)
   \tkzDefMidPoint(A,D)  \tkzGetPoint{F}
   \tkzDefMidPoint(B,C)  \tkzGetPoint{E}
   \tkzDefMidPoint(B,D)  \tkzGetPoint{Q}
   \tkzDefTangent[from = B](F,A) \tkzGetPoints{G}{H}
   \tkzInterLL(F,G)(C,D) \tkzGetPoint{J}
   \tkzInterLL(A,J)(F,E) \tkzGetPoint{K}
   \tkzDefPointBy[projection=onto B--A](K)
	 \tkzGetPoint{M}
   \tkzFillPolygon[color = green](A,B,C,D)
   \tkzFillCircle[color = orange](B,A)
   \tkzFillCircle[color = blue!50!black](M,A)
   \tkzFillCircle[color = purple](E,B)
   \tkzFillCircle[color = yellow](K,Q)
\end{tikzpicture}
\end{tkzexample}

\subsection{Clipping a disc}

\begin{NewMacroBox}{tkzClipCircle}{\oarg{local options}\parg{A,B} or \parg{A,r}}%
\begin{tabular}{lll}%
\toprule
arguments           & example & explication                         \\
\midrule
\TAline{\parg{A,B} or \parg{A,r}}{\parg{A,B} or \parg{A,2cm}} {AB radius or diameter }
\bottomrule
\end{tabular}

\medskip
\begin{tabular}{lll}%
options             & default & definition                         \\
\midrule
\TOline{radius} {radius}{circle characterized by two points defining a radius}
\TOline{R} {radius}{circle characterized by a point and the measurement of a radius }
\bottomrule
\end{tabular}

\medskip
It is not necessary to put \tkzname{radius} because that is the default option.
\end{NewMacroBox}

 \subsubsection{Example}
\begin{tkzexample}[latex=6cm,small]
  \begin{tikzpicture}
  \tkzInit[xmax=5,ymax=5]
  \tkzGrid \tkzClip
  \tkzDefPoint(0,0){A}
  \tkzDefPoint(2,2){O}
  \tkzDefPoint(4,4){B}
  \tkzDefPoint(6,6){C}
  \tkzDrawPoints(O,A,B,C)
  \tkzLabelPoints(O,A,B,C)
  \tkzDrawCircle(O,A)
  \tkzClipCircle(O,A)
  \tkzDrawLine(A,C)
  \tkzDrawCircle[fill=red!20,opacity=.5](C,O)
\end{tikzpicture}
\end{tkzexample}


\subsection{Giving a label to a circle}
\begin{NewMacroBox}{tkzLabelCircle}{\oarg{local options}\parg{A,B}\parg{angle}\marg{label}}%
\begin{tabular}{lll}%
options             & default & definition                         \\
\midrule
\TOline{radius}  {radius}{circle characterized by two points defining a radius}
\TOline{R} {radius}{circle characterized by a point and the measurement of a radius }
\bottomrule
\end{tabular}

\medskip
You don't need to put \tkzname{radius} because that's the default option. We can use the styles from \TIKZ. The label is created and therefore "passed" between braces.
\end{NewMacroBox}

\subsubsection{Example}
\begin{tkzexample}[latex=5cm,small]
\begin{tikzpicture}
 \tkzDefPoint(0,0){O} \tkzDefPoint(2,0){N}
 \tkzDefPointBy[rotation=center O angle 50](N)
     \tkzGetPoint{M}
 \tkzDefPointBy[rotation=center O angle -20](N)
      \tkzGetPoint{P}
 \tkzDefPointBy[rotation=center O angle 125](N)
      \tkzGetPoint{P'}
 \tkzLabelCircle[above=4pt](O,N)(120){$\mathcal{C}$}
 \tkzDrawCircle(O,M)
 \tkzFillCircle[color=blue!20,opacity=.4](O,M)
 \tkzLabelCircle[R,draw,fill=orange,%
       text width=2cm,text centered](O,3 cm)(-60)%
          {The circle\\ $\mathcal{C}$}
 \tkzDrawPoints(M,P)\tkzLabelPoints[right](M,P)
\end{tikzpicture}
\end{tkzexample}

\endinput
