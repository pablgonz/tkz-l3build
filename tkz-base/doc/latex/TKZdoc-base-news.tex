\section{News and presentation}

This package is the foundation of the \tkzNamePack{tkz-euclide} and \tkzNamePack{tkz-fct} in particular. It provides a Cartesian system that will be defined by the macro \tkzcname{tkzInit}. The package has been modified and object transfers between \tkzimp{tkz-base} and \tkzimp{tkz-euclide} have been performed. 
In the future, the definition macros will be isolated.

The main novelty is the recent replacement of the \tkzNamePack{fp} package by \tkzNamePack{xfp}. The appearance of this one is a step towards version 3 of \LATEX.
 The next step will be the creation of a new package.

Here are some of the changes. The |tkz-euclide| package brings more new features.
\vspace{2cm}
 \begin{itemize}\setlength{\itemsep}{10pt} 
\item Code Improvement;
\item  Bug correction;
\item  The bounding box is now controlled in each macro (hopefully) to avoid the use of \tkzcname{tkzInit} followed by \tkzcname{tkzClip};
\item Addition of macros for the  "bounding box": \tkzcname{tkzSaveBB} \tkzcname{tkzClipBB};
\item  Logically most macros accept \TIKZ\ options. So I removed the "duplicate" options;
\item  Removing the option "label options";
\item Random points are now in \tkzimp{tkz-euclide} and the macro \tkzcname{tkzGetRandPointOn} is replaced by \tkzcname{tkzDefRandPointOn}. For homogeneity reasons, the points must be retrieved with \tkzcname{tkzGetPoint};
\item The options \tkzimp{end} and \tkzimp{start} which allowed to give a label to a line are removed. You must now use the macro \tkzcname{tkzLabelLine};

\item Introduction of the libraries \NameLib{quotes} and \NameLib{angles} they allows to give a label to a point.even if I am not in favour of this practice;

\item Appearance of the macro \tkzcname{usetkztool}, which allows to load new "tools".
\end{itemize}

\endinput